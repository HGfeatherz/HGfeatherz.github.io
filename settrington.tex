
#setwd("C:/Users/HGFea/Documents/R/Markdown/New folder/settrington")

\begin{center}

\Huge{Settrington Beck: Plight of the Water Vole}

\end{center}

-------------------------------------------------------------------------------------------------------------------------------------------

This is a document outline best practice, mal-practice and routine endeavours to save the critically endangered water vowel in Settrington Beck


-------------------------------------------------------------------------------------------------------------------------------------------

\begin{center}

 Proposal

\end{center}

------
Target Species: *Carex Pendula*, "Pendulous Sedge". 
------

*C. pendula* is a non-invasive Sedge found throughout Europe and other regions, typically found in ancient woodland, water-courses and on heavy, clay soils [1]. It is noted by the RHS that *C. pendula* can be problematic as it easily escapes domestic gardens, spreading in areas with favorable conditions, such as those found within Settrington Beck and the wider catchment area. [2].




Climate change is a significant driver of change for freshwater courses [4]. Biodiversity, morphology, hydrology and  anthropogenic pressure can amount to alarmingly frequent changes to the behavior of fresh water systems as well as the ability of these systems to provide critical ecosystem services. In the context of this assessment, changes in climate are synonymous with changes in vegetative biodiversity and species prevalence. 

The benefits of Freshwater vegetation are well understood;  both *in-situ* and 'on-bank' plant growth increase sediment stabilisation, increase habitat provision and demonstrate critical nutrient filtration capabilities, allowing mitigation of impactful events such as nitrification and excess Phosphate deposition. Other broad-scale factors such as increased shading and linked reduction in evapotransipration with consequential implications for replenishment of aquifers and down-stream ecosystem service provision demonstrate the critical necessity for freshwater vegetation [6].

However, excessive vegetative growth and mono-species dominance can lead to a reduction in water quality, habitat provision and overall ecosystem health, affecting both the ecosystem services provided as well as negatively impacting the wider catchment area. This is of critical importance when considering the recent evidence outlined in the publication of the 'State of Nature Report 2023'[7].

\newpage

\begin{center}

\Large{Settrington Beck Overview}

\end{center}




In the case of Settrington Beck, a relatively rare chalk stream, previously known for its population of water voles [5]. With a total length of 11km and catchment area of some 32^2km [3], the target species has become heavily established along both Bank and *in-situ*. The issue caused by the establishment of *C. pendula* is a noticeable reduction in water flow, significant narrowing of the water-way and consequently higher vegetative growth within the stream body itself. The potential implications of reduced flow and habitat changes due to the presence of the target species outline the case for preventative measures & early treatments to mitigate significant disruptive events associated with the spreading of this vegetation.

As a feed directly into the Derwent, a significant river already heavily affected by the pervasive and non-native *Himalayan bolsom*, efforts to further mitigate the introduction of species should be considered a high-priority. 




\begin{center}

\Large{Removal of species}

\end{center}



Mechanical removal of *C. pendula* through forking or pulling has proved ineffective  due to density, location and the physical morphology of the plant. There is also concern that due to the embedded nature of the rhizomes from which *C. pendula* grows, there will be significant damage to both bank and stream bed. This damage could release large amounts of sediment and nutrients into the beck with the potential to cause nitrification and pH changes downstream, as well as alteration to the hydrological action of the beck.

The proposal here is therefor to use chemical inputs to remove the target species. This will be achieved through the use of herbicidal methods. Due to the limitations in selective herbicides for the target species, the most effective active ingredient to use would be a non-selective, systemic product. Glyphoshate would therefore represent the most practical method of target-species removal. The dangers however of using such a non-selective product can impact both invertebrate species and plant life [8]. It is therefore mission-critical a technician of appropriate qualifications is used to mitigate negative impact.


\newpage

\begin{center}

\Large{Herbicide Application}

\end{center}


Spraying would take place at specific locations to minimize herbicide impact on non-target vegetation outlined below (Figure 1). Identified regions will receive a recommended dose (per manufactures instructions) for application by a registered spray technician. Spraying will be avoided in regions of dense mixed-species sites.


```{r, out.width="49%", fig.cap="Map of target area, areas of dense target species are highlighted in red.", fig.align='center', fig.show='hold'}
knitr::include_graphics(c("C:/Users/HGFea/Documents/Pics/Untitled.png", "C:/Users/HGFea/Documents/Pics/Beck Map - Road Bridge to Brick Bridge 2.png"))
```

The application of herbicide will be performed by Dr Featherstone, possessing a PhD in Ecology, a Masters in Environmental science and significant experience in herbicide application (Trial Agronomist; PA1, PA6, PA9, PA11) and full time resident of Settrington village. Application will be performed with a simple back-pack sprayer.  



| I.D. | Area ($m^2$) | 
|------|-------------:|
|  A   | 8            |
|  B   | 22           |
|  C   | 14.5         |
|  D   | 21.5         |
|  E   | 8.7          |
|  F   | 13           |
|  G   | 2            |
|  H   | 2            |
|  I   | 6            |
| *Total*| 97.7         |

Table: Area to be sprayed


```{r, out.width="49%", fig.cap="Images of C. pendula in-situ. The significant size and proliferation of the target species dominates sections of the beck, with potential implications for the health of the beck.", fig.align='center', fig.show='hold'}
knitr::include_graphics(c("C:/Users/HGFea/Documents/Pics/sedge1.jpg", "C:/Users/HGFea/Documents/Pics/sedge2.jpg"))
```


\newpage

\begin{center}

\Large{References}


\end{center}




[^1]. https://www.woodlandtrust.org.uk/trees-woods-and-wildlife/plants/grasses-and-sedges/pendulous-sedge/ 

[^2]. https://www.rhs.org.uk/weeds/pendulous-sedge

[^3]. https://environment.data.gov.uk/catchment-planning/WaterBody/GB104027067750

[^4]. Jiao, W., Wang, L., Smith, W.K. et al. Observed increasing water constraint on vegetation growth over the last three decades. Nat Commun 12, 3777 (2021). https://doi.org/10.1038/s41467-021-24016-9

[^5]. https://www.wildtrout.org/assets/reports/Settrington%20Beck%20AV.PDF

[6] Reid, W. V., & Raudsepp-Hearne, C. (2005). Millennium ecosystem assessment.

[7] https://stateofnature.org.uk/wp-content/uploads/2023/09/TP25999-State-of-Nature-main-report_2023_FULL-DOC-v12.pdf

[8] Matozzo, V., Fabrello, J., & Marin, M. G. (2020). The effects of glyphosate and its commercial formulations to marine invertebrates: a review. Journal of Marine Science and Engineering, 8(6), 399.